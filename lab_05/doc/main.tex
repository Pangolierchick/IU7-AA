\input{style.tex}

\title{Lab 02 report}
\author{Kirill}

\date{\today}

\begin{document}
\thispagestyle{empty}

\noindent \begin{minipage}{0.15\textwidth}
    \includegraphics[width=\linewidth]{b_logo}
\end{minipage}
\noindent\begin{minipage}{0.85\textwidth}\centering
    \textbf{Министерство науки и высшего образования Российской Федерации}\\
    \textbf{Федеральное государственное бюджетное образовательное учреждение высшего образования}\\
    \textbf{«Московский государственный технический университет имени Н.Э.~Баумана}\\
    \textbf{(национальный исследовательский университет)»}\\
    \textbf{(МГТУ им. Н.Э.~Баумана)}
\end{minipage}

\noindent\rule{16cm}{3pt}
\newline\newline
\noindent ФАКУЛЬТЕТ $\underline{\text{«Информатика и системы управления»}}$ \newline\newline
\noindent КАФЕДРА $\underline{\text{«Программное обеспечение ЭВМ и информационные технологии»}}$\newline


\begin{center}
    \noindent\begin{minipage}{1.3\textwidth}\centering
    \Large\textbf{   ~~~ Лабораторная работа №5}\newline
    \textbf{по дисциплине "Анализ Алгоритмов"}\newline\newline\newline
    \end{minipage}
\end{center}

\noindent\textbf{Тема} $\underline{\text{Конвейерная обработка данных}}$\newline\newline
\noindent\textbf{Студент} $\underline{\text{Рядинский К. В.}}$\newline\newline
\noindent\textbf{Группа} $\underline{\text{ИУ7-53Б}}$\newline\newline
\noindent\textbf{Преподаватель} $\underline{\text{Волкова Л. Л.}}$\newline

\begin{center}
    \mbox{}
    \vfill
    Москва
\end{center}

\begin{center}
    \the\year ~г.
\end{center}
\clearpage

\renewcommand\contentsname{\hfill{\normalfont{СОДЕРЖАНИЕ}}\hfill}  %Оглавление
\tableofcontents
\newpage

\anonsection{Введение}

Параллельные вычисления часто используются для увеличения скорости выполнения
программ. Однако приемы, применяемые для однопоточных машин, для
параллельных могут не подходить. Конвейерная обработка данных является
популярным приемом при работе с параллельными машинами.

Целью данной работы является изучение и реализация метода конвейерных вычислений.

В рамках выполнения работы необходимо решить следующие задачи:

\begin{enumerate}
    \item изучения основ конвейерной обработки данных;
    \item применение изученных основ для реализации конвейерной обработки данных;
    \item получения практических навыков;
    \item получение статистики выполнения программы;
    \item описание и обоснование полученных результатов;
    \item выбор и обоснование языка программирования, для решения данной задачи.
\end{enumerate}

\section{Аналитическая часть}
Конвейер - способ организации вычислений, используемый в современных процессорах и контроллерах с целью повышения их производительности (увеличения числа инструкций, выполняемых в единицу времени — эксплуатация параллелизма на уровне инструкций), технология, используемая при разработке компьютеров и других цифровых электронных устройств. \\
Конвейеризация (или конвейерная обработка) в общем случае основана на разделении подлежащей исполнению функции на более мелкие части, называемые ступенями, и выделении для каждой из них отдельного блока аппаратуры. Производительность при этом возрастает благодаря тому, что одновременно на различных ступенях конвейера выполняются несколько команд.\\

Для данной работы был выбран алгоритм сборки компьютера, состоящий из трех этапов: установка материнской платы (м.п.), установка центрального процессора (цп), установка графического процессора (гп).

\subsection{Оценка производительности идеального конвейера}

Пусть задана операция, выполнение которой разбито на n последовательных этапов. При последовательном их выполнении операция выполняется за время

\begin{equation}\label{form:way} 
 \tau _{e}={\sum\limits_{i=1}^n \tau _{i}}
 \end{equation}
 \begin{align*}
    \text{где} \\
    n &- \text{количество последовательных этапов;} \\
   \tau _{i} &- \text{время выполнения i-го этапа;}
\end{align*}

Быстродействие одного процессора, выполняющего только эту операцию, составит

\begin{equation}\label{form:way} 
 S_{e}={\frac{1}{\tau _{e}}}={\frac{1}{\sum\limits_{i=1}^n \tau _{i}}}
 \end{equation}
 \begin{align*}
    \text{где} \\
    \tau _{e} &- \text{время выполнения одной операции;} \\
    n &- \text{количество последовательных этапов;} \\
   \tau _{i} &- \text{время выполнения i-го этапа;}
\end{align*}


Максимальное быстродействие процессора при полной загрузке конвейера составляет



Число n — количество уровней конвейера, или глубина перекрытия, так как каждый такт на конвейере параллельно выполняются n операций. Чем больше число уровней (станций), тем больший выигрыш в быстродействии может быть получен.

Известна оценка
\begin{equation}\label{form:way} 
{\frac{n}{n/2} \leq {\frac{S_{max}}{S_{e}}} \leq n}
 \end{equation}
 \begin{align*}
    \text{где} \\
    S_{max} &- \text{максимальное быстродействие процессора  при полной загрузке конвейера;} \\
    S_{e} &- \text{стандартное быстродействие процессора;} \\
   n &- \text{количество этапов.}
\end{align*}

то есть выигрыш в быстродействии получается от $\frac{n}{2}$  до $n$ раз.

Реальный выигрыш в быстродействии оказывается всегда меньше, чем указанный выше, поскольку:

\begin{enumerate}
\item[1)] некоторые операции, например, над целыми, могут выполняться за меньшее количество этапов, чем другие арифметические операции. Тогда отдельные станции конвейера будут простаивать;
\item[2)] при выполнении некоторых операций на определённых этапах могут требоваться результаты более поздних, ещё не выполненных этапов предыдущих операций. Приходится приостанавливать конвейер;
\item[3)] поток команд(первая ступень) порождает недостаточное количество операций для полной загрузки конвейера.
\end{enumerate}


\subsection{Вывод}

В данном разделе были рассмотрены
основополагающие материалы, которые в дальнейшем потребуются
при реализации конвейера.

\section{Конструкторская часть}

В данном разделе будут рассмотрены схемы работы конвейеров, используемые типы данных и структура программного обеспечения.

\subsection{Схемы алгоритмов}

На схеме \ref{img:main} представлены этапы запуска конвейеров.
На схеме \ref{img:worker} представлена работа одного из конвейеров.

\begin{figure}[H]
    \centering
    \includegraphics[scale=1]{main.png}
    \caption{Схема запуска конвейеров}
    \label{img:main}
\end{figure}

\begin{figure}[H]
    \centering
    \includegraphics[scale=0.95]{worker.png}
    \caption{Схема работы одного конвейера}
    \label{img:worker}
\end{figure}

\subsection{Описание используемых типов данных}

При реализации алгоритмов будут использованы следующие структуры данных:

\begin{enumerate}
    \item Конвейер ~---~ очередь.
    \item Время установки компонента ~---~ целочисленный тип long long.
    \item Номер компьютера ~---~ целочисленный тип long long.
\end{enumerate}

\subsection{Структура программного обеспечения}

Программное обеспечение состоит из следующих модулей:

\begin{enumerate}
    \item main.cpp ~---~ модуль, содержащий код точки входа.
    \item queue.h  ~---~ модуль, содержащий код реализации очереди.
    \item dns.cpp  ~---~ модуль, содержащий код конвейера.
    \item computer.h ~---~ модуль, содержащий код класса компьютера.
\end{enumerate}

\subsection{Вывод}

В данном разделе были рассмотрены схемы алгоритма, описаны используемые типы данных, структура программного обеспечения.

\section{Технологическая часть}

В данном разделе приведены средства реализации и листинги кода.

\subsection{Средства реализации}

К языку программирования выдвигаются следующие требования:

\begin{enumerate}
    \item Возможность порождать системные потоки.
    \item Возможность производить замер времени выполнения части программы.
    \item Существуют среды разработки для этого языка.
\end{enumerate}

По этим требованиям был выбран язык программирования C++.

\subsection{Листинги кода}

\begin{lstlisting}[caption=Реализация очереди, label=list:queue, language={}]
#pragma once

#include <queue>
#include <mutex>
#include <condition_variable>

template <class T>
class SafeQueue
{
public:
    SafeQueue(void) : q(), m(), c() {}
    ~SafeQueue(void) {}

    bool empty() const {
        std::lock_guard<std::mutex> lock(m);
        return q.empty();
    }

    size_t size() const {
        std::lock_guard<std::mutex> lock(m);
        return q.size();
    }
    

    void enqueue(T t)
    {
        std::lock_guard<std::mutex> lock(m);
        q.push(t);
        c.notify_one();
    }

    T dequeue(void)
    {
        std::unique_lock<std::mutex> lock(m);
        while (q.empty()) {
            c.wait(lock);
        }

        T val = q.front();
        q.pop();
        return val;
    }
private:
    std::queue<T> q;
    mutable std::mutex m;
    std::condition_variable c;
};	
\end{lstlisting}

\begin{lstlisting}[caption=Класс компьютера, label=list:computer, language={}]
#pragma once

#include <chrono>
#include <thread>


class Computer {
public:
    Computer() = default;

    bool is_built() const { return mother_board && cpu && gpu; }
    
    bool mb_installed() const { return mother_board; }
    bool cpu_installed() const { return cpu; }
    bool gpu_installed() const { return gpu; }

    void set_id(size_t i) { id = i; }
    size_t get_id() const { return id; }

    void install_mb()  { std::this_thread::sleep_for(mb_installation_time);  mother_board = true; }
    void install_cpu() { std::this_thread::sleep_for(cpu_installation_time); cpu = true; }
    void install_gpu() { std::this_thread::sleep_for(gpu_installation_time); gpu = true; }


    long long mb_install_time_s;
    long long mb_install_time_e;
    long long cpu_install_time_s;
    long long cpu_install_time_e;
    long long gpu_install_time_s;
    long long gpu_install_time_e;

private:
    bool mother_board;
    bool cpu;
    bool gpu;

    size_t id;

    const std::chrono::microseconds mb_installation_time  {1500};
    const std::chrono::microseconds cpu_installation_time {1000};
    const std::chrono::microseconds gpu_installation_time {1100};
};
\end{lstlisting}

\begin{lstlisting}[caption=Работа конвейеров, label=list:conveyor, language={}]
void dns::run_parallel() {
    std::vector<Computer> comps(ncomps);

    for (size_t i = 1; i <= comps.size(); i++) {
        comps[i - 1].set_id(i);
    }

    SafeQueue<Computer*> mb_man;
    SafeQueue<Computer*> cpu_man;
    SafeQueue<Computer*> gpu_man;

    std::atomic_bool mb_done  = false;
    std::atomic_bool cpu_done = false;

    auto qs = std::chrono::high_resolution_clock::now();


    auto mb_worker = [qs, &mb_done, &m = mb_man, &c = cpu_man]() {
        while (!m.empty()) {
            auto comp = m.dequeue();
            auto s = std::chrono::high_resolution_clock::now();
            comp->install_mb();
            auto e = std::chrono::high_resolution_clock::now();

            comp->mb_install_time_s = std::chrono::duration_cast<std::chrono::microseconds>(s - qs).count();
            comp->mb_install_time_e = std::chrono::duration_cast<std::chrono::microseconds>(e - qs).count();

            c.enqueue(comp);
        }

        mb_done = true;   
    };

    auto cpu_worker = [qs, &cpu_done, &mb_done, &c = cpu_man, &g = gpu_man]() {
        while (1) {
            if (!c.empty()) {
                auto comp = c.dequeue();
                auto s = std::chrono::high_resolution_clock::now();
                comp->install_cpu();
                auto e = std::chrono::high_resolution_clock::now();

                comp->cpu_install_time_s = std::chrono::duration_cast<std::chrono::microseconds>(s - qs).count();
                comp->cpu_install_time_e = std::chrono::duration_cast<std::chrono::microseconds>(e - qs).count();

                g.enqueue(comp);
            } else if (mb_done) {
                break;
            }
        }

        cpu_done = true;
    };

    auto gpu_worker = [qs, &cpu_done, &mb_done, &g = gpu_man]() {
        while (1) {
            if (!g.empty()) {
                auto comp = g.dequeue();
                auto s = std::chrono::high_resolution_clock::now();
                comp->install_gpu();
                auto e = std::chrono::high_resolution_clock::now();

                comp->gpu_install_time_s = std::chrono::duration_cast<std::chrono::microseconds>(s - qs).count();
                comp->gpu_install_time_e = std::chrono::duration_cast<std::chrono::microseconds>(e - qs).count();
            } else if (cpu_done && mb_done){
                break;
            }
        }
    };


    for (auto &&c : comps) {
        mb_man.enqueue(&c);
    }

    qs = std::chrono::high_resolution_clock::now();

    auto mb_t  = std::thread(mb_worker);
    auto cpu_t = std::thread(cpu_worker);
    auto gpu_t = std::thread(gpu_worker);


    mb_t.join();
    cpu_t.join();
    gpu_t.join();

    auto qe = std::chrono::high_resolution_clock::now();

    std::cout << "End time: " << std::chrono::duration_cast<std::chrono::microseconds>(qe - qs).count() << "\n";

    print_comps(comps);
}
\end{lstlisting}

\subsection{Вывод}

В данном разделе был разработан конвейер.

\section {Исследовательский раздел}

В данном разделе будет проведен замер временных характеристик выполнения алгоритмов и пример работы программы.

\subsection{Пример работы программы}

\begin{figure}[H]
    \centering
    \includegraphics[scale=0.7]{ex_run.png}
    \caption{Пример работы программы}
    \label{img:example_run}
\end{figure}


\subsection{Технические характеристики}

Технические характеристики электронно-вычислительной машины, на которой выполнялось тестирование:

\begin{itemize}
    \item операционная система: macOS BigSur версия 11.4;
    \item оперативная память: 8 гигабайт LPDDR4;
    \item процессор: Apple M1.
\end{itemize}

\subsection{Время работы программы}

В таблице \ref{tab:time} представлен лог работы программы. Лента 1 занимается установкой материнской платы, лента 2 --- установкой центрального процессора, лента 3 --- установка графического процессора. Время указано в микросекундах. \\

\begin{table}[H]
    \caption{\centering Лог работы программы}
    \centering
    \begin{tabular}{|c|c|c|c|}
    \hline
    \multicolumn{1}{|l|}{Лента №} & \multicolumn{1}{l|}{Задача №} & \multicolumn{1}{l|}{Время начала} & \multicolumn{1}{l|}{Время конца} \\ \hline
    1                             & 1                             & 40                                & 1951                             \\ \hline
    2                             & 1                             & 1966                              & 3221                             \\ \hline
    3                             & 1                             & 3227                              & 4403                             \\ \hline
    1                             & 2                             & 1962                              & 3865                             \\ \hline
    2                             & 2                             & 3876                              & 5132                             \\ \hline
    3                             & 2                             & 5137                              & 6516                             \\ \hline
    1                             & 3                             & 3873                              & 5781                             \\ \hline
    2                             & 3                             & 5789                              & 7044                             \\ \hline
    3                             & 3                             & 7059                              & 8439                             \\ \hline
    1                             & 4                             & 5789                              & 7669                             \\ \hline
    2                             & 4                             & 7672                              & 8927                             \\ \hline
    3                             & 4                             & 8932                              & 10314                            \\ \hline
    1                             & 5                             & 7671                              & 9554                             \\ \hline
    2                             & 5                             & 9559                              & 10816                            \\ \hline
    3                             & 5                             & 10817                             & 12200                            \\ \hline
    \end{tabular}
    \label{tab:time}
\end{table}

На рисунке \ref{img:plot} представлен график зависимости времени работы реализации конвейеров от количества задач для линейной и параллельной обработки конвейера.

\begin{figure}[H]
    \centering
    \includegraphics[scale=0.8]{plot.png}
    \caption{Зависимость времени работы реализации конвейеров от количества задач}
    \label{img:plot}
\end{figure}

\subsection{Вывод}

Параллельная реализация конвейерной обработки значительно выигрывает по времени относительно линейной реализации. Как видно из рисунка \ref{img:plot}, линейная реализация примерно в 2 раза медленнее параллельной при 500 задачах.


\anonsection{Заключение}

В результате выполнения лабораторной работы было экспериментально подтверждено различие временных характеристик последовательной и параллельной реализации конвейерной обработки данных.

В рамках выполнения работы была достигнута цель и решены следующие задачи:

\begin{enumerate}
	\item изучили освоили конвейерную обработку данных;
	\item применили изученные основы для реализации конвейерной обработки данных;
	\item получили практические навыки;
	\item получили статистику выполнения программы;
	\item описали и обосновали полученные результаты;
	\item выбрали и обосновали языка программирования, для решения данной задачи.
\end{enumerate}

\anonsection{Литература}

\begin{enumerate}
	\item Visual Studio Code [Электронный ресурс], режим доступа: https://code.visualstudio.com/ (дата обращения: \today)
	\item LPDDR4 [Электронный ресурс] \url{https://ru.wikipedia.org/wiki/LPDDR#LPDDR4} (дата обращения: \today)
	\item Ульянов М. В. Ресурсно-эффективные компьютерные алгоритмы. Разработка и Анализ. - Наука Физматлит, 2007. - 376.
\end{enumerate}

\end{document}

\end{document}
